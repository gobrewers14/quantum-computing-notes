\documentclass{article}

\usepackage[preprint]{./template/neurips_2020}

\usepackage[utf8]{inputenc} % allow utf-8 input
\usepackage[T1]{fontenc}    % use 8-bit T1 fonts
\usepackage{hyperref}       % hyperlinks
\usepackage{url}            % simple URL typesetting
\usepackage{booktabs}       % professional-quality tables
\usepackage{amsfonts}       % blackboard math symbols
\usepackage{nicefrac}       % compact symbols for 1/2, etc.
\usepackage{microtype}      % microtypography
\usepackage{physics}
\usepackage{amsmath}


\title{Chapter 2: Introduction to Quantum Mechanics}

\author{
  John Martinez \\
  \texttt{john.r.martinez14@gmail.com} \\
  % examples of more authors
  % \And
  % Coauthor \\
  % Affiliation \\
  % Address \\
  % \texttt{email} \\
  % \AND
  % Coauthor \\
  % Affiliation \\
  % Address \\
  % \texttt{email} \\
  % \And
  % Coauthor \\
  % Affiliation \\
  % Address \\
  % \texttt{email} \\
  % \And
  % Coauthor \\
  % Affiliation \\
  % Address \\
  % \texttt{email} \\
}

\begin{document}
\maketitle

%\begin{abstract}
%  The abstract paragraph should be indented \nicefrac{1}{2}~inch (3~picas) on
%  both the left- and right-hand margins. Use 10~point type, with a vertical
%  spacing (leading) of 11~points.  The word \textbf{Abstract} must be centered,
%  bold, and in point size 12. Two line spaces precede the abstract. The abstract
%  must be limited to one paragraph.
%\end{abstract}
\section{Intro to Quantum Mechanics}
\subsection{Linear Algebra}
Vector space
\begin{center}
  $\begin{bmatrix}
    z_{1} \\
    \vdots \\
    z_{n}
  \end{bmatrix}$
\end{center}

%%% Section 1.1.1
\subsubsection{Bases and Linear Independence}
A \emph{spanning set} for a vector space is a set of vectors
$\ket{v_{1}},...,\ket{v_{n}}$ such that any vector $\ket{v}$ in the vector
space can be written as a linear combination
$\ket{v} = \displaystyle\sum_{i}a_{i}\ket{v_{i}}$ in that set. The spanning
set for the vector space $\mathbb{C}^{2}$ is the set
  \begin{center}
    $\ket{v_{1}} \equiv \begin{bmatrix} 1 \\ 0 \end{bmatrix};\mspace{10mu}
    \ket{v_{2}} \equiv \begin{bmatrix} 0 \\ 1 \end{bmatrix}$
  \end{center}

A second spanning set in $\mathbb{C}^{2}$ is
  \begin{center}
    $
    \ket{v_{1}} \equiv \frac{1}{\sqrt{2}}
      \begin{bmatrix} 1 \\ 1
      \end{bmatrix};\mspace{10mu}
    \ket{v_{2}} \equiv \frac{1}{\sqrt{2}}
      \begin{bmatrix} 1 \\ -1
      \end{bmatrix}$
  \end{center}

A set set of non-zero vectors $\ket{v_{1}},...,\ket{v_{n}}$ are
\emph{linear dependent} if there exists a set of complex numbers
$a_{1},...,a_{n}$ with $a_{i} \neq 0$ for at least one value of $i$, such that
\begin{center}
  $a_{1}\ket{v_{1}} + a_{2}\ket{v_{2}} + \hdots + a_{n}\ket{v_{n}} = 0$.
\end{center}

%%% Section 1.1.2
\subsubsection{Linear Operators and Matrices}
A \emph{linear operator} between vector spaces $V$ and $W$ is defined to be any
function $A : V \rightarrow W$ which is linear in its inputs
  \begin{center}
    $A\left(\displaystyle\sum_{i}a_{i}\ket{v_{i}}\right) =
    \displaystyle\sum_{i}a_{i}A(\ket{v_{i}})$
  \end{center}

%%% Section 1.1.3
\subsubsection{Pauli Matrices}
  \begin{center}
    $
    \sigma_{0} \equiv I \equiv
      \begin{bmatrix} 1 & 0 \\
      0 & 1 \end{bmatrix};\mspace{10mu}
    \sigma_{1} \equiv \sigma_{x} \equiv X \equiv
      \begin{bmatrix} 0 & 1 \\
      1 & 0 \end{bmatrix}
    $
  \end{center}

  \begin{center}
    $
    \sigma_{2} \equiv \sigma_{y} \equiv Y \equiv
      \begin{bmatrix} 0 & -i \\
      i & 0 \end{bmatrix};\mspace{10mu}
    \sigma_{3} \equiv \sigma_{z} \equiv Z \equiv
      \begin{bmatrix} 1 & 0 \\
      0 & -1 \end{bmatrix}
    $
  \end{center}

%%% Section 1.1.4
\subsubsection{Inner Products}
An \emph{inner product} is a function which takes as input two vectors
$\ket{v}$ and $\ket{w}$ form a vector space and produces a complex number
as output.

A function from $V \times V \rightarrow \mathbb{C}$ is an inner product if it
satisfies the requiments that:

(1) (., .) is linear in the second argument
  \begin{center}
    $
      \left(\ket{v}, \displaystyle\sum_{i}\lambda_{i}\ket{w_{i}}\right) = 
      \displaystyle\sum_{i}\lambda_{i}(\ket{v}, \ket{w_{i}})
    $
  \end{center}

(2) $(\ket{v}, \ket{w}) = (\ket{w}, \ket{v})^{*}$

(3) $(\ket{v}, \ket{v}) \geq 0$ with equality if and only if $\ket{v} = 0$

For example, $\mathbb{C}^{n}$ has an inner product defined by
  \begin{center}
    $
      ((y_{1},...,y_{n}),(z_{1},...,z_{n})) \equiv
      \displaystyle\sum_{i}y_{i}^{*}z_{i} =
      \begin{bmatrix}
      y_{1}^{*} \hdots y_{n}^{*}
      \end{bmatrix}
      \begin{bmatrix}
      z_{1} \\
      \vdots \\
      z_{n}
      \end{bmatrix}
    $
  \end{center}

Vectors $\ket{v}$ and $\ket{v}$ are \emph{orthogonal} if their inner product
is zero. We define the \emph{norm} of a vector $\ket{v}$ by
  \begin{center}
    $\|\ket{v}\| \equiv \sqrt{\bra{v}\ket{v}}$
  \end{center}

A \emph{unit vector} is a vector $\ket{v}$ such that $\|\ket{v}\| = 1$.

Suppose $\ket{w_{1}},\hdots,\ket{w_{d}}$ is a basis set for some vector space
$V$ with an inner product. There is a useful method, the \emph{Gram-Schmidt}
procedure, which can be used to produce an orthonormal basis set
$\ket{v_{1}},\hdots,\ket{v_{d}}$ for a vector space $V$. Define
$\ket{v_{1}} \equiv \frac{\ket{w_{1}}}{\|\ket{w_{1}}\|}$ and for
$1 \leq k \leq d-1$ define $\ket{v_{k+1}}$ inductively by
  \begin{center}
  $
    \ket{v_{k+1}} \equiv
    \frac{\ket{w_{k+1}} - \sum_{i=1}^{k}\bra{v_{i}}\ket{w_{k+1}}\ket{v_{i}}}{
    \|\ket{w_{k+1}} - \sum_{i=1}^{k}\bra{v_{i}}\ket{w_{k+1}}\ket{v_{i}}\|}
  $
  \end{center}

%%% Section 1.1.5
\subsubsection{Eigenvectors and Eigenvalues}
An \emph{eigenvector} of a linear operator $A$ on a vector space is a non-zero
vector $\ket{v}$ such that $A\ket{v} = v\ket{v}$, where $v \in \mathbb{C}$ and
is known as the \emph{eigenvalue} of $A$ corresponding to $\ket{v}$. The
\emph{characteristic function} is defined as
$c(\lambda) \equiv det |A - \lambda I|$ where det is the \emph{determinant}
function of matrices. The solutions to the characteristic equation,
$c(\lambda) = 0$, are the eigenvalues of the operator $A$.

A \emph{diagonal representation} for an operator $A$ on a vector space $V$ is
a representation
  \begin{center}
    $A = \displaystyle\sum_{i}\lambda_{i}\ket{i}\bra{i}$
  \end{center}
where the vectors $\ket{i}$ form an orthonormal set of eigenvectors for $A$.
A diagonal representation of Pauli $Z$ matrix:
  \begin{center}
    $Z = \begin{bmatrix} 1 & 0 \\ 0 & -1\end{bmatrix} =
    \ket{0}\bra{0} - \ket{1}\bra{1}$
  \end{center}

%%% Section 1.1.6
\subsubsection{Adjoints and Hermitian Operators}
Suppose $A$ is any linear operator on a Hilbert space, $V$, then
$\exists A^{\dagger}$ on $V$ such that $\forall \ket{v}, \ket{w} \in V$,
  \begin{center}
    $(\ket{v}, \ket{w}) = (A^{\dagger}\ket{v}, \ket{w})$.
  \end{center}

This linear operator is know ad the \emph{adjoint} or \emph{Hermitian conjugate}
of the operator $A$.

Note: $(AB)^{\dagger} = B^{\dagger}A^{\dagger}$

($\dagger$) is called "dagger" and is equal to
$A^{\dagger} \equiv (A^{*})^{T}$, which is the transpose of the complex
conjugate. Example:
  \begin{center}
    $
    \begin{bmatrix}
      1 + 3i & 2i \\
      1 + i & 1 - 4i
    \end{bmatrix}^{\dagger} = 
    \begin{bmatrix}
      1 - 3i & 1 - i \\
      -2i & 1 + 4i
    \end{bmatrix}
    $
  \end{center}

%%% Section 1.1.7
\subsubsection{Tensor Products}
The \emph{tensor product} if a way of putting vector spaces together to form
larger vector spaces. Suppose $V$ and $W$ are vector spaces of dimensions $m$
and $n$ respectively, then $V \otimes W$ is an $nm$ dimensional vector space.
The elements of $V \otimes W$ are linear combinations of tensor products
$\ket{v}\otimes\ket{w}$ of elements $\ket{v}$ of $V$ and $\ket{w}$ of $W$. A
more intuitive way to represent this is through the \emph{Kronecker product}.
Suppose A is an $m$ by $n$ matrix and B is a $p$ by $q$ matrix then:
  \begin{center}
  $ A \otimes B = 
  \begin{bmatrix}
    A_{11}B & A_{12}B & \cdots & A_{1n}B \\
    A_{21}B & A_{22}B & \cdots & A_{2n}B \\
    \vdots  & \vdots  & \ddots & \vdots  \\
    A_{m1}B & A_{m2}B & \cdots & A_{mn}B
  \end{bmatrix}$
  \end{center}
produces an $nq \cross mp$ matrix.

Example:
  \begin{center}
    $
    \begin{bmatrix}
      1 \\ 2
    \end{bmatrix} \otimes
    \begin{bmatrix}
      2 \\ 3
    \end{bmatrix} = 
    \begin{bmatrix}
      2 \\ 3 \\ 4 \\ 6
    \end{bmatrix}
    $.
  \end{center}

%%% Section 1.1.8
\subsubsection{Operator Functions}
Let $A = \sum_{a}a\ket{a}\bra{a}$ be a spectral decomposition for a normal
operator $A$. Define $f(A) \equiv \sum_{a}f(a)\ket{a}\bra{a}$.

Example:
  \begin{center}
  $e^{\theta Z} =
    \begin{bmatrix}
      e^{\theta} & 0 \\
      0 & e^{-\theta}
    \end{bmatrix}$
  \end{center}

Since $Z$ has eigenvectors $\ket{0}$ and $\ket{1}$.

The \emph{trace} of $A$ is defined to be the sum of its diagonal elements,
  \begin{center}
    $tr(A) \equiv \displaystyle\sum_{i} A_{ii}$.
  \end{center}
Some important properities:

The trace is \emph{cyclic}: 
  \begin{center}$tr(AB) = tr(BA)$\end{center}

The trace is \emph{linear}:
  \begin{center}$tr(A + B) = tr(A) + tr(B)$ and $tr(zA) = z tr(A)$\end{center}

The trace is invariant under the \emph{unitary similarity transform}:
  \begin{center}$A \rightarrow UAU^{\dagger}$\end{center}
as
  \begin{center}$tr(UAU^{\dagger}) = tr(U^{\dagger}UA) = tr(A)$\end{center}

Suppose $\ket{\psi}$ is a unit vector and $A$ is an arbitrary operator. To
evaluate $tr(A\ket{\psi}\bra{\psi})$ use the Gram-Schmidt procedure to extend
$\ket{\psi}$ to an orthonormal basis $\ket{i}$ which includes $\ket{\psi}$ as
the first element. Then we have

  \begin{center}
    $
      tr(A\ket{\psi}\bra{\psi}) =
      \displaystyle\sum_{i}\bra{i}A\ket{\psi}\bra{\psi}\ket{i}
    $
  \end{center}
  \begin{center}
    $
      = \bra{\psi}A\ket{\psi}
    $
  \end{center}

%%% Section 1.1.9
\subsubsection{The Commutator and Anti-Commutator}
The \emph{commutator} between two operators $A$ and $B$ is defined to be
  \begin{center}
    $[A, B] \equiv AB - BA$.
  \end{center}

If $[A, B] = 0$, that is, $AB = BA$, then we say that $A$ \emph{commutes} with 
$B$.  The \emph{anti-commutator} is defined as:
  \begin{center}
    $\{A, B\} \equiv AB + BA$
  \end{center}
And we say $A$ \emph{anti-commutes} with $B$ if $\{A, B\} = 0$.

%%% Section 1.1.10
\subsubsection{Polar and Singular Value Decomposition}
Let $A$ be a linear operator on a vector space $V$. $\exists$ a unitary $U$
and positive operators $J$ and $K$ such that $A = UJ = KU$, defined by
$J \equiv \sqrt{A^{\dagger}A}$ and $K \equiv \sqrt{AA^{\dagger}}$.

Let $A$ be a square matrix. Then $\exists$ unitary matrics $U$ and $V$, and a
diagonal matrix $D$ with non-negative entries such that $A = UDV$. The
diagonal elements $D$ are called the \emph{singular values} of $A$.

\subsection{Postulates of Quantum Mechanics}
\subsubsection{State Space}
\textbf{Postulate 1}: Assocaited to any isolated physical system is a complex
vector space with inner product (that is, Hilbert space) know as the
\emph{state space} of the system. The system is completely described by its
\emph{state vector}, which is a unit vector in the system's state space.

The simplest quantum mechanical system is the \emph{qubit}. An arbitrary
vector in the state space can be written $\ket{\psi} = a\ket{0} + b\ket{1}$
where $a, b \in \mathbb{C}$. The conditional that $\ket{\psi}$ be a unit
vector, $\bra{\psi}\ket{\psi} = 1$, is equivalent to
$|a|^{2} + |b|^{2} = 1$

\subsubsection{Evolution}
How does the state, $\ket{\psi}$, of a quantum mechanical system change with
time?

\textbf{Postulate 2.1}: The evolution of a \emph{closed} quantum system is
described by a \emph{unitary transformation}. That is, the state
$\ket{\psi}$ of the system at time $t_{1}$ is related to the state
$\ket{\psi'}$ of the system at time $t_{2}$ by a unitary operator
$U$ which depends only on the times $t_{1}$ and $t_{2}$:
  \begin{center}
    $\ket{\psi'} = U\ket{\psi}$.
  \end{center}

\textbf{Postulate 2.2}: The time evolution of the state of a closed quantum
system is described by the \emph{Schrödinger equation}.

  \begin{center}
    $i\hbar\frac{\partial \ket{\psi}}{\partial t} = H\ket{\psi}$
  \end{center}

\subsubsection{Quantum Measurement}
\textbf{Postulate 3}: Quantum measurements are describled by a collection
${M_{m}}$ of \emph{measurement operators}.

If the state of the quantum system is $\ket{\psi}$ immediately before the
measurement then the probability that the result $m$ occurs is give by
  \begin{center}
    $p(m) = \bra{\psi}M_{m}^{\dagger}M_{m}\ket{\psi}$
  \end{center}
and the state after the measurement is
  \begin{center}
    $\frac{M_{m}\ket{\psi}}{\sqrt{\bra{\psi}M_{m}^{\dagger}M_{m}\ket{\psi}}}$
  \end{center}

\subsubsection{Distinguishing Quantum States}
\end{document}
