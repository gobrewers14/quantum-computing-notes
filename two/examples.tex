\documentclass{article}

\usepackage[preprint]{./template/neurips_2020}

\usepackage[utf8]{inputenc} % allow utf-8 input
\usepackage[T1]{fontenc}    % use 8-bit T1 fonts
\usepackage{hyperref}       % hyperlinks
\usepackage{url}            % simple URL typesetting
\usepackage{booktabs}       % professional-quality tables
\usepackage{amsfonts}       % blackboard math symbols
\usepackage{nicefrac}       % compact symbols for 1/2, etc.
\usepackage{microtype}      % microtypography
\usepackage{amssymb}
\usepackage{physics}
\usepackage{amsmath}

\title{Chapter 2: Introduction to Quantum Mechanics Examples}

\author{
  John Martinez \\
  \texttt{john.r.martinez14@gmail.com} \\
  % examples of more authors
  % \And
  % Coauthor \\
  % Affiliation \\
  % Address \\
  % \texttt{email} \\
  % \AND
  % Coauthor \\
  % Affiliation \\
  % Address \\
  % \texttt{email} \\
  % \And
  % Coauthor \\
  % Affiliation \\
  % Address \\
  % \texttt{email} \\
  % \And
  % Coauthor \\
  % Affiliation \\
  % Address \\
  % \texttt{email} \\
}

\begin{document}
\maketitle

%\begin{abstract}
%  The abstract paragraph should be indented \nicefrac{1}{2}~inch (3~picas) on
%  both the left- and right-hand margins. Use 10~point type, with a vertical
%  spacing (leading) of 11~points.  The word \textbf{Abstract} must be centered,
%  bold, and in point size 12. Two line spaces precede the abstract. The abstract
%  must be limited to one paragraph.
%\end{abstract}
\section{Intro to Quantum Mechanics}
\subsection{Linear Dependence}
Show that $(1, -1), (1, 2)$ and $(2, 1)$ are linearly dependent.

\textbf{Solution}

The solution to the linear system is $x = y = -z$.  So let
$x = y = 2, z = -2$ then $ 2(1, -1) + 2(1, 2) - 2(2, 1) = (0, 0)\blacksquare$

\subsection{Matrix Representations}
Suppose $V$ is a vector space with basis vectors $\ket{0}$ and $\ket{1}$, and
$A$ is a linear operator from $V$ to $V$ such that $A\ket{0} = \ket{1}$
and $A\ket{1} = \ket{0}$. Give a matrix representation for $A$, with respect
to the input basis $(\ket{0}, \ket{1})$ and the output basis
$(\ket{0}, \ket{1})$.

\textbf{Solution}

$\sigma_{x} \equiv X \equiv \begin{bmatrix} 0 & 1 \\ 1 & 0\end{bmatrix}$ will
flip a qubit in the standard basis. Let $\ket{0} =
\begin{bmatrix}1 \\ 0\end{bmatrix}$ and let $\ket{1} = 
\begin{bmatrix}0 \\ 1\end{bmatrix}$ then $X\ket{0} =
\begin{bmatrix}0 & 1 \\ 1 & 0\end{bmatrix}\begin{bmatrix} 1 \\ 0\end{bmatrix} = 
\begin{bmatrix}0 \\ 1\end{bmatrix}$ and $X\ket{1} =
\begin{bmatrix}0 & 1 \\ 1 & 0\end{bmatrix}\begin{bmatrix} 0 \\ 1\end{bmatrix} =
\begin{bmatrix}1 \\ 0\end{bmatrix}\blacksquare$

\subsection{Matrix Representation for Operator Products}
Suppose $A$ is a linear operator from vector space $V$ to vector space $W$,
and $B$ is a linear operator from vector space $W$ to vector space $X$.
Let $\ket{v_{i}}, \ket{w_{j}}$ and $\ket{x_{k}}$ be bases for the vector
spaces $V$, $W$, and $X$, respectively. Show that the matrix
representation for the linear transformation $BA$ is the matrix product
of the matrix representation for $B$ and $A$, with respect to the appropriate
bases.

\textbf{Solution}
$\blacksquare$


\end{document}

