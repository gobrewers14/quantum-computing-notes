\documentclass{article}

\usepackage[preprint]{./template/neurips_2020}

\usepackage[utf8]{inputenc} % allow utf-8 input
\usepackage[T1]{fontenc}    % use 8-bit T1 fonts
\usepackage{hyperref}       % hyperlinks
\usepackage{url}            % simple URL typesetting
\usepackage{booktabs}       % professional-quality tables
\usepackage{amsfonts}       % blackboard math symbols
\usepackage{nicefrac}       % compact symbols for 1/2, etc.
\usepackage{microtype}      % microtypography
\usepackage{physics}
\usepackage{amsmath}


\title{Chapter 1: Introduction and Overview}

\author{
  John Martinez \\
  \texttt{john.r.martinez14@gmail.com} \\
  % examples of more authors
  % \And
  % Coauthor \\
  % Affiliation \\
  % Address \\
  % \texttt{email} \\
  % \AND
  % Coauthor \\
  % Affiliation \\
  % Address \\
  % \texttt{email} \\
  % \And
  % Coauthor \\
  % Affiliation \\
  % Address \\
  % \texttt{email} \\
  % \And
  % Coauthor \\
  % Affiliation \\
  % Address \\
  % \texttt{email} \\
}

\begin{document}
\maketitle

%\begin{abstract}
%  The abstract paragraph should be indented \nicefrac{1}{2}~inch (3~picas) on
%  both the left- and right-hand margins. Use 10~point type, with a vertical
%  spacing (leading) of 11~points.  The word \textbf{Abstract} must be centered,
%  bold, and in point size 12. Two line spaces precede the abstract. The abstract
%  must be limited to one paragraph.
%\end{abstract}

\section{Introduction}

%%% Section 1.1
\subsection{Global Perspective}
n/a

%%% Section 1.2
\subsection{Quantum Bits}
What is a qubit? Just as a classical bit has a \emph{state} - either $0$ or $1$
- a qubit also has a state. Two possible states for a qubit are the states
$\ket{0}$ and $\ket{1}$, which as you might guess corresponds to the states
$0$ and $1$ for a classical bit. The difference between bits and qubits is
that a qubit can be in a state \emph{other} than $\ket{0}$ or $\ket{1}$. It
is also possible to form \emph{a linear combination} of states, often
called \emph{superpositions}:

\begin{center}
  $\ket{\psi} = \alpha\ket{0} + \beta\ket{1}$
\end{center}

Where $\alpha, \beta \in \mathbb{C}$.  Unfortunately, we cannot examine a qubit
to determine its quantum state, that is, the values of $\alpha$ and
$\beta$.  Instead, quantum mechanics tells us that we can only aquire
much more restricted information about the quantum state. When we measure a
qubit we get either the result $0$, with probability $\abs{\alpha}^{2}$, or the
result $1$, with probability $\abs{\beta}^{2}$ with
$\abs{\alpha}^{2} + \abs{\beta}^{2} = 1$, since probabilities must sum to $1$.

This equation can also be represented as 

\begin{center}
  $\ket{\psi} = \cos(\frac{\theta}{2})\ket{0} + e^{i\phi}\sin(\frac{\theta}{2})\ket{1}$
\end{center}

%%% Section 1.2.1
\subsubsection{Multiple Qubits}
A two qubit system will have four \emph{computational basis states} denoted
$\ket{00}, \ket{01}, \ket{10}, \ket{11}$. The state vector describing two qubits is

\begin{center}
  $\ket{\psi} = \alpha_{00}\ket{00} + \alpha_{01}\ket{01} + \alpha_{10}\ket{10} + \alpha_{11}\ket{11}$
\end{center}
And again, the probabilities must sum to one, satisfying the condition
$\sum_{x \in \{0, 1\}^{2}}{\abs{\alpha_{x}^{2}} = 1}$, where the notation
'$x \in \{0, 1\}^{2}$' means 'the set of strings of length two with each letter
being either zero or one'.  Subsets of multi-qubit systems can be measured.
Measuring the first qubit of the above two qubit system gives $0$ with
probability $\abs{\alpha_{00}^{2}} + \abs{\alpha_{01}^{2}}$. The
post-measurement state would then be

\begin{center}
    $\ket{\psi'} = \frac{\alpha_{00}\ket{00} + \alpha_{01}\ket{01} }{\sqrt{\abs{\alpha_{00}^{2}} + \abs{\alpha_{01}^{2}}}}$
\end{center}

An important two qubit state is the \emph{Bell state} or \emph{EPR pair}, 

\begin{center}
    $\frac{\ket{00} + \ket{11}}{\sqrt{2}}$
\end{center}

Bell states are important because their measurement outcomes are
\emph{correlated}.

%%% Section 1.3
\subsection{Quantum Computation}

%%% Section 1.3.1
\subsubsection{Single Qubit Gates}
The quantum $\mathbf{NOT}$ gate acts \emph{linearly}.
\begin{center}
  $\mathit{X} \equiv \begin{bmatrix}
     1 & 0 \\
     0 & 1
   \end{bmatrix}$
\end{center}
If we write the quantum state $\ket{\psi} = \alpha\ket{0} + \beta\ket{1}$ in
vector notation
\begin{center}
    $\begin{bmatrix}
       \alpha \\
       \beta
     \end{bmatrix}$
\end{center}
then the $\mathbf{NOT}$ gate operation can be represented as
\begin{center}
  $\mathit{X} \begin{bmatrix}
     \alpha \\
     \beta
   \end{bmatrix} = \begin{bmatrix}
     \beta \\
     \alpha
   \end{bmatrix}$
\end{center}
There is a restriction on the the type of matrix that can be a quantum gate and
that is that the operation must preserve the normalization condition (i.e.
$\abs{\alpha}^{2} + \abs{\beta}^{2} = 1$). The appropriate condition on the
matrix representing the gate is that the matrix $\mathit{U}$ describing the
single qubit gate must be \emph{unitary}, that is
$\mathit{U}^{\dagger}\mathit{U} = \mathit{I}$ (where $\mathit{X}^{\dagger}$
is the transpose of the complex conjugate of $\mathit{X}$).

Other important single qubit quantum gates are the $\mathit{Z}$ gate
\begin{center}
  $\mathit{Z} \equiv \begin{bmatrix}
    1 & 0 \\
    0 & -1
  \end{bmatrix}$
\end{center}
which leaves $\ket{0}$ unchanged and flips the sign of $\ket{1}$, and the
\emph{Hadamard} gate
\begin{center}
  $\mathit{H} \equiv \frac{1}{\sqrt{2}}\begin{bmatrix}
    1 & 1 \\
    1 & -1
  \end{bmatrix}$
\end{center}
The Hadamard operation is just a rotation of the sphere about the $\hat{y}$
axis by $90^{\circ}$, followed by a rotation about the $\hat{x}$ axis by
$180^{\circ}$.

%%% Section 1.3.2
\subsubsection{Multiple Qubit Gates}
The prototypical multi-qubit logic gate is the \emph{controlled}-$\mathbf{NOT}$
or $\mathbf{CNOT}$ gate.

\begin{center}
  $ \mathit{U_{\mathit{CN}}} \equiv \begin{bmatrix}
    1 & 0 & 0 & 0 \\
    0 & 1 & 0 & 0 \\
    0 & 0 & 0 & 1 \\
    0 & 0 & 1 & 0
  \end{bmatrix}$
\end{center}
The gate has two inputs, a \emph{control} qubit and a \emph{target} qubit. If
the control qubit is set to $0$, then the target qubit is left alone. If the
control qubit is set to $1$, then the target qubit is flipped.

\begin{center}
  $\ket{00} \to \ket{00}$;
  $\ket{01} \to \ket{01}$;
  $\ket{10} \to \ket{11}$;
  $\ket{11} \to \ket{10}$;
\end{center}

%%% Section 1.3.3
\subsubsection{Measurement in non-standard basis}
The states $\ket{0}$ represent just one of many possible choices of basis
states for a qubit. Another possible choice is

\begin{center}
  $\ket{+} \equiv \frac{(\ket{0} + \ket{1})}{\sqrt{2}} $ and
  $\ket{-} \equiv \frac{(\ket{0} - \ket{1})}{\sqrt{2}} $
\end{center}

%%% Section 1.3.4
\subsubsection{Quantum Circuits?}
%%% Need to learn how to draw circuits in LaTex

%%% Section 1.3.5
\subsubsection{Copying Circuit?}

%%% Section 1.3.6
\subsubsection{Example: Bell States}

\begin{center}
  $\ket{\beta_{00}} = \frac{\ket{00} + \ket{11}}{\sqrt 2}$;

  $\ket{\beta_{01}} = \frac{\ket{01} + \ket{10}}{\sqrt 2}$;

  $\ket{\beta_{10}} = \frac{\ket{00} - \ket{11}}{\sqrt 2}$;

  $\ket{\beta_{11}} = \frac{\ket{01} - \ket{10}}{\sqrt 2}$;
\end{center}
These are known as the \emph{Bell states}, or \emph{EPR states or pairs}.

%%% Section 1.4
\subsection{Quantum Algorithms}

%%% Section 1.4.1
\subsubsection{Classical Computations on a Quantum Computer?}

%%% Section 1.4.2
\subsubsection{Quantum Parallelism}
Suppose $f(x) : \{0, 1\} \to \{0, 1\}$ is a function with a one-bit domain
and range. A convenient way of computing this function on a quantum computer is
to consider a two qubit quantum computer which starts in the state $\ket{x, y}$.
It is possible with a sequence of gates to turn this into the state
$U_f : \ket{x, y} \to \ket{x, y \oplus f(x)}$.  In particular, if we apply
the Hadamard gate to
$\mathit{H} \ket{0}$ we obtain $\frac{\ket{0} + \ket{1}}{\sqrt 2}$. Now
applying
$U_f$ we get $\frac{\ket{y, y \oplus f(0)} + \ket{y, y \oplus f(1)}}{\sqrt 2}$.
And now letting $y = \ket{0}$, we get
$\frac{\ket{0, f(0)} + \ket{0, f(1)}}{\sqrt 2}$.  This is a remarkable state,
in that we have information about $f(0)$ and $f(1)$ simultaneously!

This procedure can me generalized to functions on an arbitrary number of bits,
by using the \emph{Walsh-Hadamard transform}. For $n = 2$ qubits we get

\begin{center}
  $\frac{\ket{0} + \ket{1}}{\sqrt 2}\frac{\ket{0} + \ket{1}}{\sqrt 2} = 
   \frac{\ket{00} + \ket{01} + \ket{10} + \ket{11}}{2}$
\end{center}
We can write $\mathit{H}^{\otimes 2}$ to denote the parallel action of two
Hadamard gates. More generally, the result of performing the Walsh-Hadamard
transform on $n$ qubits initially in the $\ket{0}$ state is

\begin{center}
  $\frac{1}{\sqrt{2^{n}}}\displaystyle\sum_{x} \ket{x}$
\end{center}
And we write $\mathit{H}^{\otimes n}$ to denote this operation.

% \begin{center}
% \end{center}

\end{document}
